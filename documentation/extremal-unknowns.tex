\documentclass[12pt]{amsart}

\usepackage[margin=1in]{geometry}
\usepackage{lmodern}
\usepackage{amsmath,amsthm,amssymb,mathrsfs}

\title{Extremal Unknowns}
\date{}
\author{Robert Sanders}

\newtheorem*{theorem}{Theorem}
\newtheorem*{lemma}{Lemma}
\newtheorem*{proposition}{Proposition}
\newtheorem*{corollary}{Corollary}

\theoremstyle{definition}
\newtheorem*{definition}{Definition}

\theoremstyle{remark}
\newtheorem*{claim}{Claim}
\newtheorem*{remark}{Remark}

\newcommand{\C}{\mathbb{C}}
\newcommand{\N}{\mathbb{N}}
\newcommand{\Q}{\mathbb{Q}}
\newcommand{\R}{\mathbb{R}}
\newcommand{\Z}{\mathbb{Z}}
\newcommand{\ra}{\rightarrow}
\newcommand{\cL}{\mathcal{L}}
\newcommand{\start}{\mathtt{start}}
\newcommand{\halt}{\mathtt{halt}}
\newcommand{\Left}{\mathsf{L}}
\newcommand{\Stay}{\mathsf{S}}
\newcommand{\Right}{\mathsf{R}}
\newcommand{\A}{\mathsf{A}}
\newcommand{\sC}{\mathsf{C}}
\newcommand{\co}{\mathsf{co}}
\newcommand{\cocap}{\mathsf{cocap}}
\newcommand{\NP}{\mathsf{NP}}
\newcommand{\sP}{\mathsf{P}}
\newcommand{\poly}{\mathsf{poly}}
\newcommand{\msf}[1]{\mathsf{#1}}
\newcommand{\id}{\msf{id}}
\newcommand{\AND}{\mathbin{\&}}
\newcommand{\OR}{\mathbin{\text{or}}}
\newcommand{\Oh}{\mathcal{O}}
\newcommand{\inner}[1]{\left\langle#1\right\rangle}
\newcommand{\op}{\msf{op}}
\newcommand{\sN}{\msf{N}}
\newcommand{\BP}{\msf{BP}}
\newcommand{\exppad}{\msf{exppad}}
\newcommand{\FP}{\msf{FP}}
\newcommand{\CZ}{\msf{CZ}}
\newcommand{\subsetequ}{\stackrel{?}{\subseteq}}

\begin{document}
\maketitle

Identifying key missing information is an important part of the survey process.
Complexity Zoology has the capacity to identify a subset of unknown inclusions
that are sufficient to decide all other unknowns. To be precise, recall that
an inclusion $\sC_1\subseteq_W\sC_2$ can exist in one of three possible states:
proven, disproven, or open. However, since Complexity Zoology's data set is
incomplete, it is possible that the system cannot infer which of the three
possibilities applies to a given inclusion. When this occurs, we say that the
inclusion is \textit{unknown}, not to be confused with the inclusion being
open. In short, to say that an inclusion is open is to say that it does not
have a known proof or disproof, while to say that an inclusion is unknown
means that Complexity Zoology itself does not know the status. We denote an
unknown inclusion by $\sC_1\subsetequ_W\sC_2$.

Ideally, we would like to complete the data set so that there are no unknown
inclusions. However, this is a process that potentially involves a great deal
of redundancy, since listing a formerly unknown inclusion as proven (for
example) may decide several other unknown inclusions. Therefore, it is helpful
to identify a subset of unknown inclusions that are sufficient to decide all
others. We refer to these as \textit{extremal unknowns}.

A \textit{most likely extremal unknown} is, roughly, an unknown inclusion that,
if it were listed as proven, would not result in any other inclusions being
listed as proven. Complexity Zoology considers extremal unknowns separately for
each of the worlds, and for simplicity inference rules involving multiple
worlds or involving operatros are not considered. Thus, the only inference
rules that are relevant to determining whether Complexity Zoology considers an
unknown inclusion to be a most likely extremal are
\begin{align*}
p(\sC_1\subseteq_W\sC_2)\AND p(\sC_2\subseteq_{W^*}\sC_3)&\Longrightarrow
p(\sC_1\subseteq_W\sC_3), \\
p(\sC_1\subseteq_W\sC_2)\AND p(\sC_2\subseteq_{W^*}\sC_3)&\Longrightarrow
p(\sC_1\subseteq_W\sC_3),
\end{align*}
where $W$ is the world under consideration and $W^*$ is the transitive dual. In
testing whether an unknown inclusion $\sC_1\subsetequ_W\sC_2$ is a most likely
extremal, Complexity Zoology tests whether there exists a third class $\sC_3$
that is distinct from $\sC_1$ and $\sC_2$ in $W$ satisfying one of the
following two conditions:
\begin{align*}
\sC_1\subsetequ_W\sC_3\AND p(\sC_2\subseteq_{W^*}\sC_3), \\
\sC_3\subsetequ_W\sC_2\AND p(\sC_3\subseteq_{W^*}\sC_1).
\end{align*}
If such a $\sC_3$ exists, then the system concludes that the unknown is
\textit{not} most likely extremal.

Similarly, a \textit{least likely extremal unknown} is intended to be an
unknown inclusion that, if listed as disproven, would not result in any other
unknown inclusions being decided as disproven. In this case, the relevant
inference rules are
\begin{align*}
d(\sC_1\subseteq_W\sC_2)\AND p(\sC_3\subseteq_{W^*}\sC_2)&\Longrightarrow
p(\sC_1\subseteq_W\sC_3), \\
d(\sC_1\subseteq_W\sC_2)\AND p(\sC_1\subseteq_{W^*}\sC_3)&\Longrightarrow
p(\sC_3\subseteq_W\sC_2).
\end{align*}
Complexity Zoology checks whether a given unknown inclusion
$\sC_1\subsetequ_W\sC_2$ is a least likely extremal by checking whether there
exists a class $\sC_3$ that is distinct from $\sC_1$ and $\sC_2$ in $W$ and
satisfies at least one of the following:
\begin{align*}
\sC_1\subsetequ_W\sC_3\AND p(\sC_3\subseteq_{W^*}\sC_2), \\
\sC_3\subsetequ_W\sC_2\AND p(\sC_1\subseteq_{W^*}\sC_3).
\end{align*}
Complexity Zoology lists extremal unknowns of both types for each world. It
also specifies whether each extremal is provable, disprovable, or completely
unknown (neither provable nor disprovable).
\begin{proposition}
Deciding the extremal unknowns is sufficient to decide all the unknowns for a
world.
\end{proposition}
The unknowns that Complexity Zoology has identified as extremal have resulted
in several interesting questions. They have also been extremely useful in
filling in large portions of the data very quickly, in particular, a disproof
for a most likely inclusion or a proof for a least likely inclusion tend to
settle many other unknowns.

\end{document}
