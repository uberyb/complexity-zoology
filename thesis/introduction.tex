This document is a description of a computer program called \textit{Complexity
Zoology}. The program is an \textit{expert system}: it is equipped with a
database of information about which complexity classes are subsets of other
complexity classes along with an inference engine that the program uses to
deduce new conclusions from the existing information. Complexity Zoology then
outputs a diagram of the relationships between the complexity classes in the
input file.

Complexity Zoology takes its name from the Complexity Zoo, an online wiki of
information about complexity classes written and maintained by Scott
Aaronson \cite{aaronsonZoo}. This version of Complexity Zoology was written from scratch; however,
some of the design choices---particularly the input syntax and the
functionality of the output diagram---are based on an earlier version by Greg
Kuperberg.

The purpose and motivation of Complexity Zoology is to be a partially automated
survey of complexity theory. The scope of a field as well-developed as
complexity theory is so large that it can be difficult to quickly determine the
status of a proposition of the form ``the complexity class $\sC_1$ is contained
in the complexity class $\sC_2$.'' Complexity Zoology aims to help with this
problem, at least for the most fundamental complexity classes. More
specifically, the project aims to:
\begin{itemize}
\item summarize a large portion of the known complexity class inclusions and
  oracle separations;
\item identify redundant results (i.e., the results that follow logically from
  other known results);
\item answer questions about complexity class relations automatically when the
  answer is a corollary of established results.
\end{itemize}
Complexity Zoology has proved itself useful in identifying when a result is a 
corollary of other results and in falsifying conjectures, but the demonstrated 
strength of the program has been the identification of stimulating open problems. 
Among the questions that the project has raised are these:
\begin{enumerate}
\item We know that $\msf{NP}\subseteq\msf{IP}\subseteq\msf{MIP}$, and these classes 
  are believed to be distinct. Is there an oracle relative to which 
  $\msf{MIP}\not\subseteq\msf{IP}$? Does $\msf{IP}=\NP$ relative to the random 
  oracle?
\item Is there an oracle relative to which 
  $\msf{NISZK}\not\subseteq\co\msf{NIQSZK}$?
\item Is there an oracle relative to which $\msf{SZK}\not\subseteq S_2\sP$?
\item Is there an oracle relative to which $\msf{BQP}\not\subseteq \msf{IP}$?
\end{enumerate}

To make Complexity Zoology an effective tool, it has been necessary to limit its
scope in a few ways. First, the system's expertise does not lie in reasoning 
about complexity theory as such. It knows nothing of the standard techniques 
used to prove results in complexity theory, such as diagonalization. It does not
even understand what complexity classes are: common classes such as $\sP$, 
$\NP$, and $\msf{BPP}$ are understood only in terms of their relationships to 
other complexity classes. Instead, the strength of Complexity Zoology consists 
of understanding results and open problems in the field. For example, if it is 
known that $\sC_1\subseteq\sC_2$ is proven and $\sC_1\subseteq\sC_3$ is an open 
question, then Complexity Zoology can conclude that $\sC_2\subseteq\sC_3$ is 
unproven -- either it has been disproven, or it is itself open. In essence, the 
system can be thought of as a diligent student conducting a broad overview of 
the field, drawing connections between results and attempting to fill in all 
possible gaps without examining the details.

Second, the project has been deliberately limited to a core collection of
important complexity classes. The program has the capacity to identify critical
gaps in its knowledge, and by choosing a conservative list of classes we
increase the chance that the questions Complexity Zoology asks are of
theoretical interest. The system's input syntax makes it easy to add and remove
classes as needed, so adjustments can be made as necessary.
