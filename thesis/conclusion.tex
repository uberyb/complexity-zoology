\section{Conclusion and Possible Future Work}

By now we have seen the power and limitations of Complexity Zoology. Along the way, also outlined the landscape of complexity theory. We discussed class operators and the role they play in the Zoology program. We explained the software's logic and how it can infer whether an inclusion is proven, disproven, or open. We explored a simplified version of the system's data set with a reduced number of complexity classes to consider. Finally, we surveyed all the classes currently in the project while highlighting some of the key definitions and results.

Just as an inquisitive student can inspire new lines of thought with an insightful question, Complexity Zoology has brought attention to some challenging but not widely studied open problems. Completing the system's data set by identifying which inclusions should be regarded as truly open and which inclusions can be marked as proven or disproven will comprise a large amount of the potential future work on the project. Another possibility for further expanding Complexity Zoology is considering a different collection of complexity classes or of different oracle worlds. Several results in complexity theory are conditioned on a possible outcome of an open problem: for example: ``If $\sP=\NP$, then..." Such results could be studied with Complexity Zoology's help by including an ``if $\sP=\NP$'' or similar world.

We have been careful to emphasize that Complexity Zoology does not understand complexity theory as such. Even so, the software could be a helpful tool in a true formalization of complexity theory based on computer-assisted of computer-verified proofs. In particular, Zoology's ability to weed out redundant inclusions or separations could be an invaluable time-saver.